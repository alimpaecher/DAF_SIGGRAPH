% The big data revolution is profoundly transforming computer science
% and society. Problems which frustrated generations of researchers,
% such as machine translation and object recognition, suddenly became
% ``solved'' when run on large datasets. However, the scarcity of data
% in many domains threatens the continued success of the big data
% approach. We now see a schism between domains for which large
% datasets are available (such as translation corpuses) and those for
% which it is not, with much more rapid progress in the former. We
% propose a method of collecting new datasets through crowd sourcing
% and social game mechanics. First, we assemble a corpus of aligned
% drawings via a new iPhone game, {\em DrawAFriend} developed
% specifically for the purpose of collecting drawing data. Second, by
% analyzing the database of drawings, we build a spatially varying
% model of artistic consensus at the stroke level. Using this model,
% we introduce a surprisingly simple method to improve strokes in
% real-time. Importantly, our auto-corrections appear nearly invisible
% to the user, while seamlessly preserving artistic intent. Lastly, we
% use the game itself to evaluate the effectiveness of our stroke
% correction algorithm. We do this by randomly turning on stroke
% correction for half our player base, and AB testing how
% auto-correction effects their drawings.

We propose a new method for the large-scale collection and analysis
of drawing through a touch-screen game developed specifically for
the purpose of collecting drawing data. Analyzing this crowdsourced
drawing database, we build a spatially varying model of artistic
consensus at the stroke level. We then present a surprisingly simple
stroke-correction method which uses our artistic consensus model to
improve strokes in real-time. Importantly, our auto-corrections run
interactively and appear nearly invisible to the user while
seamlessly preserving artistic intent. Closing the loop, the game
itself serves as a platform for large-scale evaluation of the
effectiveness of our stroke correction algorithm.
