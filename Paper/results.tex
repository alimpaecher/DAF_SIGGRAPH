\section{Results}
\subsection{DrawAFriend Game Stats}
On January 8th 2013 the DrawAFriend game officially launched. In just 4 days later, we had already collected 6373 drawings. Of that dataset we choose 611 drawings that were of extremely high quality to be used for analysis. \alex{This may conflict or repeat previous mentions of the high quality subset} The following statistics the drawings at this point in time. 

Overall the DrawAFriend game was sucessful right out the gate. It was downloaded by over 2000 players \alex{2114 players if we want to be specific}, creating 2191 games, and drawings 6373 drawings. 

When drawing players drew for about X seconds in the entire drawing database, and Y drawings in the quality drawings. All in all players spend X \# hours simply drawings in DrawAFriend.

\subsubsection{Celebrity Stats}
Players have the options of drawing Facebook Friends or celebrties when playing DrawAFriend. They had the choice to draw Robert Downey Jr., Angelina Jolie, Kim Kardashian, Barack Obama, Brad Pitt, or Kristen Stewart. The stats in the X table referenc the 611 super high quality drawings.

Average \# of Strokes:
Average Length of Strokes:
Average Length of Guessing: 
Percent Correct and Incorrect: 

\subsection {Correction Vector Field}
The mean shift solver takes X amoutn of time 
Take a look at the figures
It was created in matlab

\subsection {Applications}
\subsubsection{Existing User Drawings}
One application of the correction vector field is improving the existing database of user drawings. The drawings are already quite good, thus making the corrections made by the vector field all the more impressive.  While the algorithm universally improves the images by making the celebrity far more recognizable, it does so without sacrificing style. For examples of the raw drawings and the corrected images please see the video and the suplemntary website.

\subsubsection{Stroke Correction for Interactive Drawings}
A second applicaiton of the correction vector field is to interactively modify strokes. Without adding any more user interface elements to the existing DrawAFriend drawing UI, we seamlessly added stroke auto-correction. As the user draws, our method subtly corrects the user stroke at interactive rates on an iPhone 4. This allows the drawer to be far less strigent when drawing a celebrity. Please see the \alex{Not sure how to reference the teaser on top} and the video for interactive examples.




\begin{figure}
\centering
\begin{tabular}{ccccc}
\imgtbl{image_aj} & \imgtbl{avg_aj} & \imgtbl{dir_aj} & \imgtbl{mag_aj} & \imgtbl{edges_aj} \\
\imgtbl{image_bp} & \imgtbl{avg_bp} & \imgtbl{dir_bp} & \imgtbl{mag_bp} & \imgtbl{edges_bp} \\
\imgtbl{image_kk} & \imgtbl{avg_kk} & \imgtbl{dir_kk} & \imgtbl{mag_kk} & \imgtbl{edges_kk} \\
\imgtbl{image_ks} & \imgtbl{avg_ks} & \imgtbl{dir_ks} & \imgtbl{mag_ks} & \imgtbl{edges_ks} \\
\imgtbl{image_rd} & \imgtbl{avg_rd} & \imgtbl{dir_rd} & \imgtbl{mag_rd} & \imgtbl{edges_rd} \\
\imgtbl{image_bo} & \imgtbl{avg_bo} & \imgtbl{dir_bo} & \imgtbl{mag_bo} & \imgtbl{edges_bo} \\
(a) & (b) & (c) & (d) & (e)
\end{tabular}
\caption{DrawAFriend users draw one of six celebrities (a). We use our database of hundreds of drawings per subject -- shown averaged in (b) -- to precompute a \emph{correction vector field} (c) enabling real-time drawing assistance on the iPhone. The magnitude of our vector field (d) reveals a consensus of artistic renderings strikingly different than what we could compute with automated methods, such as a canny edge detector (e).}
\label{fig:image-table}
\end{figure}
% a test edit by Michael
