\section{Related Work}

Overall our work resonates with the ideas proposed in Surowiecki's \emph{Wisdom of the Crowds} \shortcite{Surowiecki:2005}. Perhaps the best known games designed to collect data are the games of von Ahn and colleagues \shortcite{vonAhn:2004:LIC,vonAhn:2006:PGL} which used games to label objects in images. By contrast, we ask players to complete the much more complex (and creative) task of actually drawing new images. The graphics community is witnessing a recent spike of interest in ``big data'' approaches to understanding drawing, with the prototypical examples being \emph{ShadowDraw} \cite{Lee:2011} which help guide freeform sketching and \emph{WhatsMySketch} \cite{Eitz:2012:HSO} which identifies iPhone sketches as one of 250 possible object classes. An important distinction between these efforts and our work is the data-collection approach. Lee \etal used 30,000 images downloaded from existing web databases and extracted edges as proxies for possible sketches. Eitz \etal collected 20,000 rasterized sketches on Amazon Mechanical Turk (https://www.mturk.com). By contrast, we created a publicly available iPhone game \emph{DrawAFriend} which uses mechanics similar to \emph{DrawSomething} (http://omgpop.com/drawsomething) to intrinsically motivate players to contribute drawings. Therefore, rather than sequestering data collection into an initial phase of our research, we collect data continuously with zero marginal cost per user, and can re-instrument the game to change data collection on the fly. An important distinction is that \emph{ShadowDraw} and \emph{WhatsMySketch} study freeform sketching, while our celebrity database consists of many registered drawings of the same image.

There has also been considerable work in the Human Computer Interaction community on solving the problem of inaccurate touch interactions, often called the \emph{fat finger problem}. Wigdor \etal \shortcite{Wigdor:2009:RUP} taxonomized touch-based interaction errors, and presented novel visual cues to help the user understand their intent. Other work has addressed the fat finger problem by designing new interaction patterns \cite{Albinsson:2003:HPT,Benko:2006:PST,Forlines06hybridpointing,Vogel07shift:a},or adding additional interaction hardware \cite{Scott:2010:RTE,Wigdor:2006:UTI,Wigdor:2007:LTS}. The book by Benko and Wigdor \shortcite{Benko:Fatfinger} presents a good overview of work in the field.

There has been little work in helping users draw. One system, \emph{iCanDraw} \cite{Dixon:2010:IUS} helps users draw faces with a tutorial approach. A sketching beautification system~\cite{Orbay2011} takes many sketchy short overlapping strokes and infers smooth lines. \emph{Elasticurves} \cite{Thiel:2011} neatened sketches by smoothing strokes dynamically based on stroke speed.  \emph{ShadowDraw} \cite{Lee:2011}, like our system, uses collected images to aid users in drawing by providing {\em shadows} of similar drawings.  In contrast to these methods and to the \emph{fat finger problem} corrections, we add no additional hardware, visual cues or interaction paradigms. The user thinks they are simply drawing with no help. We use data from a corpus of previous drawings of the same image to seamlessly correct user strokes as they draw. Gingold's work\shortcite{Gingold:2012} used image averaging to beautify images. However their method, which involved averaging euclidean points, was only used to improve simple drawings such as smiley faces. Our stroke correction method was inspired by the finding of Cole \etal \shortcite{Cole:2008:WDP} that artists frequently draw similar lines. Our results confirm Cole's finding, and further develop this idea with our hypothesis that the consensus of these strokes represents the fundamental user ``intent.'' Our method detects and leverages this artistic consensus in order to
interactively adjust strokes.

Our stroke-correction technique minimizes an energy function similar
in spirit to the intelligent scissors
method \cite{Mortensen:1995:ISF}. However, rather than snap to image
contours, our energy function is based on a consensus of strokes from many
registered drawings. We further preserve the {\em style} of the
users' stroke geometry using a Gradient method which is often used
in geometry processing \cite{Botsch:2008:LVS}. 