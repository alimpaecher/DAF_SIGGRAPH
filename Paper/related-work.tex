\section{Related Work}

The graphics community is witnessing a recent spike of interest in
``big data'' approaches to understanding drawing, with the
prototypical examples being \emph{ShadowDraw} \cite{Lee:2011} which
help guide freeform sketching and \emph{WhatsMySketch}
\cite{Eitz:2012:HSO} which identifies iPhone sketches as one of 250
possible object classes. A important distinction between these
efforts and our work is the data-collection approach. Lee \etal used
30,000 sketches downloaded from existing web databases in raster
format which Eitz \etal collected 20,000 sketches on Amazon
Mechanical Turk (https://www.mturk.com). By contrast created a
publicly available iPhone game \emph{DrawAFriend} which uses
mechanics similar to \emph{DrawSomething}
(http://omgpop.com/drawsomething) to intrinsically motivate players
to contribute drawings. Therefore, rather than sequestering data
collection into an initial phase of our research, we collect data
continuously with zero marginal cost per user, and can re-instrument
the game to change data collection on the fly. Our game approach
also gives us more information per user, including knowledge of
their social graph, but requires us to design a compelling game to
collect data. In this sense, \daf is closer to the games of von Ahn
and colleagues \shortcite{vonAhn:2004:LIC,vonAhn:2006:PGL} which
uses games to label objects in images. By contract, we ask players
to complete the much more complex (and creative) task of actually
drawing new images. Another important distinction is that \emph{ShadowDraw} and
\emph{WhatsMySketch} study freeform sketching, while our celebrity
database consists of many registered drawings of the same image.

There has also been considerable work in the Human Computer
Interaction community on solving the problem of inaccurate touch
interactions, often called the \emph{fat finger problem}. Wigdor
\etal \shortcite{Wigdor:2009:RUP} taxonomized touch-based
interaction errors, and presented novel visual cues to help the user
understand their intent. Other work has addressed the fat finger
problem by designing new interaction patterns
\cite{Albinsson:2003:HPT,Benko:2006:PST,Forlines06hybridpointing,Vogel07shift:a},
or adding additional interaction hardware
\cite{Scott:2010:RTE,Wigdor:2006:UTI,Wigdor:2007:LTS}. The book by
Benko and Wigdor \shortcite{Benko:Fatfinger} presents a good
overview of work in the field.

By contrast, we add no additional hardware, visual cues or
interaction paradigms. Instead we use data from lots of previous
drawings of the same image to seamlessly correct user strokes as
they draw. Our stroke correction approach draws inspiration from the
work of Cole \etal \shortcite{Cole:2008:WDP}, which studied
collected statistics on a small number of hand-collected, registered
sketches of the same object. Our stroke correction method was
inspired by the finding of Cole \etal (confirmed by our data) that
artists frequently draw similar lines. We further develop this this
idea with our hypothesis that the average of these strokes
represents the fundamental user ``intent'' towards which we snap
strokes. Our approach minimizes an energy function on the stroke
similar to the intelligent scissors method
\cite{Mortensen:1995:ISF}. However, rather than snap to image
contours \adrien{Note: it would be interesting to have this as a
comparison.}, our energy function is based on average strokes from
many registered sketches. We further preserve stroke geometry using
a Laplacian method which is often used in geometry processing
\cite{Sorkine:2004:LSE}. Another system called \emph{iCanDraw}
\cite{Dixon:2010:IUS} helps users draw faces, although using a
tutorial approach rather than looking at automatic correction based
on a large dataset.

% 
% [ ] Intelligent Scissors :: 
% 	[ ] Intelligent Scissors (a.k.a. Live Wire or Magnetic Lasso)
% 	[Mortensen and Barrett 1995] allows a user to choose a “minimum
% 	cost contour” by roughly tracing the object’s boundary with the
% 	mouse. A
% [ ] Laplacian Mesh Editting ~
% 
