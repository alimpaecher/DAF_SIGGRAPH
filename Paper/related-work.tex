\section{Related Work}

The graphics community is witnessing a recent spike of interest in
``big data'' approaches to understanding drawing, with the
prototypical examples being \emph{ShadowDraw} \cite{Lee:2011} which
help guide freeform sketching and \emph{WhatsMySketch}
\cite{Eitz:2012:HSO} which identifies iPhone sketches as one of 250
possible object classes. A important distinction between these
efforts and our work is the data-collection approach. Lee \etal used
30,000 sketches downloaded from existing web databases in raster
format which Eitz \etal collected 20,000 sketches on Amazon
Mechanical Turk (https://www.mturk.com). By contrast created a
publicly available iPhone game \emph{DrawAFriend} which uses
mechanics similar to \emph{DrawSomething}
(http://omgpop.com/drawsomething) to intrinsically motivate players
to contribute drawings. In principle our game approach means that we
have ``never ending'' data collection which can be re-instrumented
on the fly in response to changing demands from the research
community, and also could prove more cost-effective than Mechanical
Turk if the game gets significant user uptake.


. The former uses a database of  . The latter uses a
database of 
Studying large databases of hand-drawn imagse

[ ] Shadow Draw ~\cite{}
	[ ] guiding the freeform drawing of objects
	[ ] database of 30,000 sketches collected from the web in (raster format)
[ ] WhatsMySketch - Eitz et all (The Amazon Turk Drawing Project in Siggraph last year) :: \cite{}
	[ ] 20,000 sketches from amazon mechanical turk

by von Ahn and co-authors. 

[ ] peekaboo - vonAhn:2006:PGL
[ ] ESP Game - vonAhn:2004:LIC

In that sense our paper is closer 

[ ] Where Do People Draw Lines? ~\cite{Cole:2008:WDP}
	[ ] analysis 
[ ] iCanDraw :: Dixon:2010:IUS
	[ ] tutorial style interface to help people draw faces better
		[ ] no big data component

Visual Cues:

[x] http://research.microsoft.com/en-us/um/people/benko/publications/2009/Ripples%20UIST09.pdf
	%http://research.microsoft.com/en-us/um/people/benko/publications/2009/Ripples_UIST.wmv
	- taxonomize touch-based interaction errors
	- adress this problem with visual cues
	- good for pointing interface, not drawing interface

New Interaction Patterns Which solve the problem (picking, dragging, standard interaction patterns)

[x] ripples 1  - High precision touch screen interaction :: Albinsson:2003:HPT:642611.642631

[c] High Precision Touchscreens: Design Strategies and Comparison with a Mouse. (**) :: Albinsson:2003:HPT:642611.642631
[c] ripples 3 - Precise selection techniques for multitouch screens. :: Benko:2006:PST:1124772.1124963
[c] ripples 8 - Hybridpointing :: Forlines06hybridpointing:fluid
[c] ripples 23 - Shift :: Vogel07shift:a

[s] Touch screens now offer compelling uses.
[s] Improving the accuracy of touch screens: an experimental evaluation of three strategies
[s] ripples 15
[s] ripples 21 - HybridTouch

Changing the hardware:

[c] ripples 11 - BehindTouch :: Scott:2010:RTE:1851600.1851630
[c] ripples 24 - Under the Table :: Wigdor:2006:UTI:1166253.1166294
[c] ripples 25 - LucidTouch :: Wigdor:2007:LTS:1294211.1294259

Study the fingertip geometry of touching:

[ ] Understanding touch :: holz2011ut
	% http://www.christianholz.net/understanding_touch.html

For a good overview please see:

[ ] book by Hrvoje Benko with Daniel Wigdor :: \cite{Benko:Fatfinger}

Deformation...

[ ] Intelligent Scissors :: Mortensen:1995:ISF
	[ ] Intelligent Scissors (a.k.a. Live Wire or Magnetic Lasso)
	[Mortensen and Barrett 1995] allows a user to choose a “minimum
	cost contour” by roughly tracing the object’s boundary with the
	mouse. A
[ ] Laplacian Mesh Editting ~\cite{Sorkine:2004:LSE}

