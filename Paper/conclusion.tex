\section{Conclusion}

DrawAFriend
a successful new iPhone game we developed specifically for the
purpose of collecting drawing data.

One idea for bigger picture: Is that we have developed a format of dealing with the ``fat finger'' problem. A data-driven method to deal with that issue. (Though this might be more appropriate for an HCI project)

Looking further into the future, we hope that DrawAFriend will establish a template for Facebook applications that incentivizes the creation of large complex datasets. This will create a virtuous cycle between user interfaces, social experiences, and research to promote our understanding of our humanity in this ever evolving interconnected world.

First, we assemble a corpus of aligned drawings via a new iPhone
game we developed specifically for the purpose of collecting drawing
data. Second, by analyzing the database of drawings, we build a
spatially varying model of artistic consensus at the stroke level.
Using this model, we introduce a surprisingly simple method to
improve strokes in real-time. Importantly, our corrections appear
nearly invisible to the user, seamlessly preserving artistic intent
while simplifying touch-based drawing.

Such stroke
information could allow us to rigorously answer questions about the
order in which artists draw lines, and how quickly.