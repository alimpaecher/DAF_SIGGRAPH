\section{Conclusion}

We have presented a unique crowdsourcing approach of using social game mechanics to grow and use a dataset of drawings. We developed an iPhone game, \emph{DrawAFriend}, specifically for the purpose of
collecting drawing data. We then introduced a method to extract
stroke-level artistic consensus from a large drawing corpus. The
resulting \emph{correction vector field} improves strokes in
real-time without new interactions interfaces, while preserving artistic
intent. We presented stroke correction on over 80 images which illustrate our method's ability to improve both
aesthetics and recognizability. Lastly we evaluated our stroke correction algorithm through crowd sourcing, by observing how it influenced drawings "in the wild" by instrumenting the game itself. In general, we believe that \daf presents an unprecedented platform to perform quantitative drawing analysis at the Internet scale.

Our stroke consensus method has several drawbacks. While it allows us to improve strokes and filter out non-portrait drawings, it currently requires the initial drawings to be filtered by hand. Automated filtering is an area of future research. Even with stroke
correction, many drawings are still not beautiful. We would like to
study automatic aesthetic enhancement of strokes, including weight
and texture. We also see several avenues to improve and generalize
our correction vector field model. For example, the field depends on
2D position only. An anisotropic field, however, could correct strokes
differently based on their orientation. We plan to {\em lift} the correction field to
3D (position plus orientation). This could be especially
beneficial at stroke intersections which we do not explicitly model
at present. Nevertheless, we have found that our simple isotropic
field corrects strokes well in practice, especially
around dominant lines for which there is great artistic consensus.
Moving beyond the single consensus model, we envision automatically
identifying and clustering by artistic ``style''. This would allow new drawings to
be enhanced based on which style they most closely mimic to provide more
nuanced stroke correction. Lastly we would like to extend our stroke correction algorithm to make it more general, perhaps by using an image-analogy-like approach to transfer our correction vector-field from one image to another.

Stroke correction represents just the tip of the iceberg for
applications of the large (and growing) \daf image corpus. Thus far, we have explored only a
small portion of the trove of data we are collecting. For example,
measuring how quickly drawings are identified could shed light on
which strokes are most salient for recognition. Studying when the
user undoes strokes could provide clues as to which strokes are
good or bad. While the stroke correction algorithm could have been implemented using just a rasterized image of a drawing, the vector representation will be useful for other applications. Analyzing stroke order could enable us to predict what
the artist will draw next, potentially enabling completely new
drawing interactions. Making this data available to the community,
we hope to explore both these exciting ideas, and discover as-yet
unknown applications of this rich dataset.

%  expect to find yet
% more exciting applications for this new drawing corpus.

%
%
% when images are guessed correctly
%
% - whether they guessed quickly and how long
% 	- use this as weights for the consensus.
% 	- which were most
%
% For example, we could study
%
% are using only a
% small portion of the data
%
%  Rigorous analysis of the data
% could provide insight into how people draw specific facial features,
% in what order do artists draw strokes, and which strokes
%
% First, we assemble a corpus of aligned drawings via a new iPhone
% game we developed specifically for the purpose of collecting drawing
% data.
%
% Such stroke
% information could allow us to rigorously answer questions about the
% order in which artists draw lines, and how quickly.
% 	
% - we want to learn a lot more from these drawings
% 	- how to people draw eyes
% 	- what order do they draw strokes
% 	- what will they draw next?
%
%
% - things that the game tells that we're not yet exploiting
% 	- how
% 	- where they zoom
%
%
% - future work
%
% 	- 