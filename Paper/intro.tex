\section{Introduction}

Drawing as a means of communication dates well before other forms of
recorded history. Even as photography has come to dominate visual
communication, drawing still holds an important role in artistic
expression. Unfortunately, two factors make drawing difficult.
Although drawing skills are very learnable, few people take the time
to master this medium. In addition, although electronic devices can
help this by providing, for example, tracing methods, the {\em fat
finger} phenomenon makes drawing on phones and tablets very error
prone. We address these problems in two ways. First we construct a
game, {\em DrawAFriend}, that encourages people to draw (in our
initial case celebrity portraits), and second we use the corpus of
drawings from previous players to aid new drawers overcome the fat
finger problem by attracting their strokes to the consensus of
previous drawings while leaving their individual styles intact. This
application represents only a first use of a new dataset of hand
drawings collected by the game.

The big data revolution is profoundly transforming computer science
and society. Problems which frustrated generations of researchers,
such as machine translation and object recognition, suddenly are on
the brink of becoming ``solved'' when run on large datasets. We now
see a schism between domains for which large datasets are available
ñ such as translation corpuses ñ and those for which it is not,
with much more rapid progress in the former.

One domain suffering from such data scarcity is hand-drawn images.
An ideal drawing corpus would have precise stroke-level data,
including timing and order. Such stroke information could allow us
to rigorously answer questions about the order in which artists draw
lines, and how quickly. We might even glean semantic knowledge about
the subject by understanding the temporal relationships among
strokes. For statistical purposes, we would like a \emph{large}
dataset, with many drawing by the same artist and many drawings of
the same subject by different artists.

To address these issues, we recently developed an iPhone game called
\emph{DrawAFriend}, which is intended to generate a corpus of
hand-drawn images containing all of the information described above.
We currently focus on face portraits. Such drawings are exceedingly
difficult to draw by hand, and even more so using a touch interface.
To aid users and to collect multiple drawings of the same subject,
we allow players to trace over existing photographs.

Our use of a game to generate data brings the added challenge that
the game must be rewarding and encourage repeated play. We address
this challenge by allowing player to draw mutual friends on the
social network as well as celebrities. We phrase the challenge as a
guessing game, with the important corollary that we can learn
\emph{when} a particular face was recognized, i.e. which strokes
were necessary for recognition. The success DrawAFriend is beginning
to appear: within DrawAFriend's first 3 days on the market,
we generated over 5,000 images, all with stroke-level information.

In addition to describing DrawAFriend, we also demonstrate a first
application of the initial drawing data to provide a self-correcting
touch-based drawing interface on mobile devices. We observe that
drawing with a touch device often suffers from the ``fat finger''
problem. We factor this issue into two elements: (1) the ``intent''
of the artist in drawing a stroke, and (2) an additional random
noise component caused by inaccuracy in the touch interface. We
therefore hypothesize that if strokes are ``average'' (in an
appropriate sense) over a sufficiently large database of drawings,
then we can cancel out the noise and recover the original intent.
Furthermore, this data allows us to develop a ``self-correcting''
touch interface: in essence, we clean up the artists drawing in real
time by using data from hundreds of previous drawings of the same
subject. To solve this problem, we present a new metric for ``local
stroke coherence,'' that is, regions of the images where many
artists agree. We further present a surprisingly simple method to
correct strokes based on this metric making it significantly easier
to draw faces. The interface appears ``invisible'' as the result
feels more like the intent of the user than the originally stroke.


%%%%%%%%%%%%%%%%%%%%%%%%%%%%%%
% Adrien's Old Section Below %
%%%%%%%%%%%%%%%%%%%%%%%%%%%%%%

% The big data revolution is profoundly transforming computer science
% and society. Problems which frustrated generations of researchers,
% such as machine translation and object recognition, suddenly became
% ``solved'' when run on large datasets. However, the scarcity of data
% in many domains threatens the continued success of the big data
% approach. We now see a schism between domains for which large
% datasets are available – such as translation corpuses – and
% those for which it is not, with much more rapid progress in the
% former.
%
% One domain suffering from such data scarcity is hand-drawn images.
% Visual expression is a fundamental human trait with deep connections
% to perception and creativity, yet most of our understanding of the
% drawing process is purely qualitative. Although a quick Google
% search yields million of line drawings, these images are stored in
% raster format with little or no semantic annotation. Given this
% state of affairs, a great many fascinating questions about human
% drawing cannot be answered quantitatively. An ideal drawing corpus
% would be annotated by artist and subject allowing us to measure
% variation in ``drawing style'' across a single subject, while
% illumination how one artist's style varies across subjects. Another
% goal would be to have precise stroke-level data, including timing
% and order. Such stroke information could allow us to rigorously
% answer questions about the order in which artists draw lines, and
% how quickly. We might even glean semantic knowledge about the
% subject by understanding the temporal relationships among strokes.
% Going further, drawings are meant to be recognized and understood by
% others. Ideally, a drawing corpus could shed light on these issues,
% for example: which strokes were most salient in conveying the
% subject? Finally, for statistical purposes, we would like a
% \emph{large} dataset, with many drawing by the same artist and many
% drawings of the same subject.
%
% To address these issues, we have developed an iPhone game called
% \emph{DrawAFriend}, which is intended to generate millions of
% hand-drawn images containing all of the information described above.
% In comparison with previous work which studied simple line drawings
% \adrien{[cite Hayes and Shadowdraw]}, we focus on face portraits.
% Such drawings are exceedingly difficult to draw by hand, and even
% more so using a touch interface. Therefore, we decided to allow
% players to trace over existing photographs. Even within this
% restricted setting, we observe startling variation in drawing style
% \adrien{maybe this should be the teaser image?}. Moreover, when
% multiple artists trace the same face, we get the added benefit that
% all drawings are in geometric correspondence, which can lead to much
% simplified analysis. In contrast to previous work which generated
% drawings through Amazon Mechanical Turk, our use of a game to
% generate data brings the added challenge that the game must be
% rewarding and encourage repeated play. We address this challenge by
% allowing player to draw mutual friends on the social network as well
% as celebrities. We phrase the challenge as a guessing game, with the
% important corollary that we can learn \emph{when} a particular face
% was recognized, i.e. which strokes were necessary for recognition.
% The success our game design is born out in the numbers: within
% DrawAFriend's first \textbf{X} days on the market, we generated
% \textbf{Y} images, all with stroke-level information.
%
% We believe that the completely new, data-rich corpus generated by
% this game will allow us to solve numerous fascinating questions,
% including those delineated above. In this paper, we focus on
% studying how such a large database can help us create
% self-correcting touch-based interactions on mobile devices. We
% observe that drawing with a touch device often suffers from the
% ``fat finger'' problem. We factor this issue into two elements: (1)
% the ``intent'' of the artist in drawing a stroke, and (2) an
% additional random noise component caused by inaccuracy in the touch
% interface. We therefore hypothesize that if strokes are ``average''
% (in an appropriate sense) over a sufficiently large database of
% drawings, then we can cancel out the noise and recover the original
% intent. Furthermore, this data can allow us to develop a
% ``self-correcting'' touch interface: in essence, we can clean up the
% artists drawing in real time by using data from hundreds of previous
% drawings of the same subject. To solve this problem, we present a
% new metric for ``local stroke coherence,'' that is, regions of the
% images where many artists agree. We further present a surprisingly
% simple method to correct strokes based on this metric making it
% significantly easier to draw faces.
%
% More generally, we consider the DrawAFriend corpus to be a
% significant contribution which will benefit research by the graphics
% community, and we hope that our game design insights might guide
% other researchers seeking to collect large datasets in domains
% suffering from data scarcity.


% - fat finger problem	
% 	- althogh we believe
% 	- with applications from automated graphic training to
% 	
%
% 	- problem with drawing using touch interfaces
% 	- naively we can use zooming but this isn't fun
% 	- how can we leverage many drawings
% 	- our insight: average strokes are good
%
% 	- this simple method can clean up many strokes,
% 	- more generally
% 		- valuable corpus which we will make available to the community
% 		- and will also yield design patterns
% 		which we can reuse to incentivize large groups to generate further
% 		datasets,
%
% To solve this problem, my research uses social networking mechanism
% to elicit large datasets that have never been assembled before. As a
% model problem, I am working on human drawing.  Nonetheless, recent work by my colleagues and I
% shed light on the power of computation to illuminate fundamental
% aspects of this process. At Princeton, I worked on a highly cited
% SIGGRAPH paper which computationally analyized line drawings. To
% achieve this result however, we painstakingly set up workshops where
% artists would draw pre-selected objects. The dataset itself only
% included 208 line drawings from 29 artists. This work inspired my
% collaborators to develop ShadowDraw, which demonstrated that large
% image datasets can be leveraged a real-time free-hand drawing
% guidance software which leverages a large database of images from
% the internet. Unsupervised learning on thousands or millions of
% drawings would be groundbreaking. The creation of this dataset was
% not possible until Facebook integration and my project, DrawAFriend.

\adrien{Note we got over 5,000 images in the first three days of the game.} 