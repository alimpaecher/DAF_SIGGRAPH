\section{Introduction}

Drawing as a means of communication dates well before other forms of
recorded history. Today, drawing remains a vital form of artistic
expression and an important window into human perception. However,
the central challenge to further scientific analysis of drawing is
data scarcity. Although search engines index a huge collection of
line drawings, these images are stored in raster format with little
or no useful metadata. Ideally, a drawing corpus would contain
precise stroke-level data for each image, including timing
information. We would also like semantic metadata identifying
artists and subjects. Even more ambitiously, we would like to glean
\emph{perceptual} information, such as which strokes
contributed most to image recognition. Finally, for statistical purposes,
we would like a \emph{large} dataset, with many drawings by the same
artist and many drawings of the same subject by different artists.

To address this challenge, we developed \emph{DrawAFriend}, an
iPhone game specifically designed to collect drawing data, including
all of the information described above. We currently focus on face
portraits. Such drawings are exceedingly difficult to draw by hand,
and even more so using a touch interface on a small mobile device.
To aid users and to collect multiple drawings of the same subject,
we allow players to trace over existing photographs. In its first
week of release \daf generated over 1,500 images per day.

We believe that this large and continuously growing drawing database
will enable a rich stream of future research in graphics. As a first
application, we demonstrate how the \daf corpus can be mined to
provide a self-correcting touch-based drawing interface on mobile
devices. We observe that drawing with a touch device often suffers
from the ``fat finger'' problem. We conceptually factor this issue
into two elements: (1) the ``intent'' of the artist in drawing a
stroke, and (2) an additional random noise component caused by
inaccuracy in the touch interface. We therefore hypothesize that if
we can determine a consensus of strokes (in an appropriate sense)
over a sufficiently large database of drawings, then we can cancel
out the noise and recover the artist's original intent. We analyze
the drawing corpus to compute a \emph{correction vector field} that
for any location, points towards a nearby consensus of strokes. This
allows us to develop a real-time {\em self-correcting} touch
interface: as artists draw, we essentially clean up their drawings by using data from previous drawings of the same subject. We
further introduce a surprisingly simple method to correct strokes
based on this consensus while maintaining the {\em stylistic}
choices of the artist. The interface requires no new user
interaction paradigms, in other words, it appears ``invisible'' to
the user. The resulting strokes feel more like the intent of the
user than the raw original strokes.

To validate the effectiveness of our auto-correction algorithm,
we ran a large-scale user study within the game. Each time a
user draws a celebrity, we randomly turn the stroke correction
on or off. Our results validate the effectiveness of our stroke
correction algorithm: with autocorrect on, artists do not need to
draw as accurately, and they undo their strokes less. The ability of
\daf to serve simultaneously as a large-scale visual data
collection platform and as a statistically relevant user study
underscores the generality of our crowdsourcing approach.

% Thus far, we have explored only a
% small portion of the trove of data we are collecting. For example, whether a drawing was guessed correctly could help determine
% which strokes are most salient for recognition. Undone strokes could help us understand which strokes are good and bad.  Furthermore, stroke order could enable us to predict what
% the artist will draw next. We hope to explore all these exciting ideas, and discover as-yet
% unknown applications of this rich dataset.

%\adrien{We need a few sentences or a paragraph hear explaining how
%we have only begun to explore the trove of information contained in
%this corpus, indicating that we conclude by sketching some further
%applications of this data. Much of this can come from the rebuttal,
%I think.}  -- this is in the rebuttal do we want it in the end

%We do actually have this at the very end is that enough? It makes sense to close off the paper like that...

% hrough the use of crowd sourcing, we develop a novel
% way of understanding how people draw. First we construct a game,
% {\em DrawAFriend}, that encourages people to draw. Second, we use
% the corpus of drawings from previous players to aid new drawers to
% overcome the fat finger problem by auto-correcting strokes by
% attracting them towards the consensus of strokes in previous
% drawings while leaving their individual styles intact. This
% application represents only a first use of a new dataset of hand
% drawings collected by the game. Lastly we use the {\em DrawAFriend}
% game itself to evaluate the effectiveness of our stroke correction
% algorithm by dividing users into two groups, with and without
% auto-correction on.
% 




% The big data revolution is profoundly transforming computer science
% and society. Problems which frustrated generations of researchers,
% such as machine translation and object recognition, suddenly are on
% the brink of becoming ``solved'' when run on large datasets. We now
% see a schism between domains for which large datasets are available
% such as translation corpuses and those for which it is not,
% with much more rapid progress in the former.
%
% One domain suffering from such data scarcity is hand-drawn images.








%Our use of a game to generate data brings the added challenge that
%the game must be rewarding and encourage repeated play. We address
%this challenge by allowing players to draw mutual friends on the
%as well as celebrities. We phrase the challenge as a
%guessing game, with the important corollary that we can learn
%\emph{when} a particular face was recognized, i.e. which strokes
%were necessary for recognition.




%%%%%%%%%%%%%%%%%%%%%%%%%%%%%%
% Adrien's Old Section Below %
%%%%%%%%%%%%%%%%%%%%%%%%%%%%%%

% The big data revolution is profoundly transforming computer science
% and society. Problems which frustrated generations of researchers,
% such as machine translation and object recognition, suddenly became
% ``solved'' when run on large datasets. However, the scarcity of data
% in many domains threatens the continued success of the big data
% approach. We now see a schism between domains for which large
% datasets are available – such as translation corpuses – and
% those for which it is not, with much more rapid progress in the
% former.
%
% One domain suffering from such data scarcity is hand-drawn images.
% Visual expression is a fundamental human trait with deep connections
% to perception and creativity, yet most of our understanding of the
% drawing process is purely qualitative.  Given this
% state of affairs, a great many fascinating questions about human
% drawing cannot be answered quantitatively. An ideal drawing corpus
% would be annotated by artist and subject allowing us to measure
% variation in ``drawing style'' across a single subject, while
% illumination how one artist's style varies across subjects. Another
% goal would be to have precise stroke-level data, including timing
% and order. Such stroke information could allow us to rigorously
% answer questions about the order in which artists draw lines, and
% how quickly. We might even glean semantic knowledge about the
% subject by understanding the temporal relationships among strokes.
% Going further, drawings are meant to be recognized and understood by
% others. Ideally, a drawing corpus could shed light on these issues,
% for example: which strokes were most salient in conveying the
% subject? Finally, for statistical purposes, we would like a
% \emph{large} dataset, with many drawing by the same artist and many
% drawings of the same subject.
%
% To address these issues, we have developed an iPhone game called
% \emph{DrawAFriend}, which is intended to generate millions of
% hand-drawn images containing all of the information described above.
% In comparison with previous work which studied simple line drawings
% \adrien{[cite Hayes and Shadowdraw]}, we focus on face portraits.
% Such drawings are exceedingly difficult to draw by hand, and even
% more so using a touch interface. Therefore, we decided to allow
% players to trace over existing photographs. Even within this
% restricted setting, we observe startling variation in drawing style
% \adrien{maybe this should be the teaser image?}. Moreover, when
% multiple artists trace the same face, we get the added benefit that
% all drawings are in geometric correspondence, which can lead to much
% simplified analysis. In contrast to previous work which generated
% drawings through Amazon Mechanical Turk, our use of a game to
% generate data brings the added challenge that the game must be
% rewarding and encourage repeated play. We address this challenge by
% allowing player to draw mutual friends on the social network as well
% as celebrities. We phrase the challenge as a guessing game, with the
% important corollary that we can learn \emph{when} a particular face
% was recognized, i.e. which strokes were necessary for recognition.
% The success our game design is born out in the numbers: within
% DrawAFriend's first \textbf{X} days on the market, we generated
% \textbf{Y} images, all with stroke-level information.
%
% We believe that the completely new, data-rich corpus generated by
% this game will allow us to solve numerous fascinating questions,
% including those delineated above. In this paper, we focus on
% studying how such a large database can help us create
% self-correcting touch-based interactions on mobile devices. We
% observe that drawing with a touch device often suffers from the
% ``fat finger'' problem. We factor this issue into two elements: (1)
% the ``intent'' of the artist in drawing a stroke, and (2) an
% additional random noise component caused by inaccuracy in the touch
% interface. We therefore hypothesize that if strokes are ``average''
% (in an appropriate sense) over a sufficiently large database of
% drawings, then we can cancel out the noise and recover the original
% intent. Furthermore, this data can allow us to develop a
% ``self-correcting'' touch interface: in essence, we can clean up the
% artists drawing in real time by using data from hundreds of previous
% drawings of the same subject. To solve this problem, we present a
% new metric for ``local stroke coherence,'' that is, regions of the
% images where many artists agree. We further present a surprisingly
% simple method to correct strokes based on this metric making it
% significantly easier to draw faces.
%
% More generally, we consider the DrawAFriend corpus to be a
% significant contribution which will benefit research by the graphics
% community, and we hope that our game design insights might guide
% other researchers seeking to collect large datasets in domains
% suffering from data scarcity.


% - fat finger problem	
% 	- althogh we believe
% 	- with applications from automated graphic training to
% 	
%
% 	- problem with drawing using touch interfaces
% 	- naively we can use zooming but this isn't fun
% 	- how can we leverage many drawings
% 	- our insight: average strokes are good
%
% 	- this simple method can clean up many strokes,
% 	- more generally
% 		- valuable corpus which we will make available to the community
% 		- and will also yield design patterns
% 		which we can reuse to incentivize large groups to generate further
% 		datasets,
%
% To solve this problem, my research uses social networking mechanism
% to elicit large datasets that have never been assembled before. As a
% model problem, I am working on human drawing.  Nonetheless, recent work by my colleagues and I
% shed light on the power of computation to illuminate fundamental
% aspects of this process. At Princeton, I worked on a highly cited
% SIGGRAPH paper which computationally analyized line drawings. To
% achieve this result however, we painstakingly set up workshops where
% artists would draw pre-selected objects. The dataset itself only
% included 208 line drawings from 29 artists. This work inspired my
% collaborators to develop ShadowDraw, which demonstrated that large
% image datasets can be leveraged a real-time free-hand drawing
% guidance software which leverages a large database of images from
% the internet. Unsupervised learning on thousands or millions of
% drawings would be groundbreaking. The creation of this dataset was
% not possible until Facebook integration and my project, DrawAFriend.

% \adrien{Note we got over 5,000 images in the first three days of the game.} 